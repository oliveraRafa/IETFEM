\documentclass{article}
% pre\'ambulo

\usepackage{lmodern}
\usepackage[T1]{fontenc}
\usepackage[spanish,activeacute]{babel}
\usepackage{mathtools}

\title{Desarrollo de una Interfaz Gr�fica para una Herramienta de C�lculo de Estructuras}
\author{Rafael Olivera - Federico Garc�a}

\begin{document}
% cuerpo del documento

\maketitle

\newpage
$\ $
\newpage
$\ $

\tableofcontents

\newpage
$\ $
\newpage
$\ $

\section {Introducci�n}

	\subsection {Definici�n del problema y motivaci�n}
	
	introduccion hablando de el problema de resolucion de estructuras
	
	hablar de que los programas comerciales son diferentes
	
	de que trata el proyecto(agregar interfa)
	
	quienes integramos el equipo
	
	se busca agregar usabilidad y efucuebcua
	
	\subsection {Desarrollo previo}
	
	inicios del ietfem, motivacion
	
	tecnologias en las que esta hecho
	
	explicar que es lo que hace
	
	uso en clases y criticas de estudiantes
	
	\subsection {Objetivos y resultados esperados}
	
	como mencionamos antes (repetir los problemas y comparar concomerciales brevemente)
	
	objetivo: mejorar la usabilidad, hablar de mejorar la entrada, visualiar los resultados, manipulacion de resultados
	
	objetivo: mejorar la eficiencia, hablar de mejorar tiempos, de que queremos que el estudiante no pierda tiempo en aspectos computacionales
	
	resultados esperados: describir brevemente el sistema: queremos un sistema agil que mejore ambos aspectos, que pueda cumplir las espectativas del curso de elasticidad, y que por lo tanto se va a probar en ejercicios del curso etc etc
	
	\subsection {Desarrollo del proyecto}
	
	Entendimiento del problema real y uso del core
	
	investigar sobre herramientas similares
	
	Fase inicial: busqueda de tecnologias
	
	Busqueda de herramientas similares
	
	decisiones sobre tecnologias, porque la elegimos brevemente, porque decidimos hacerlo eb
	
	definion de funcionalidades y alcance
	
	programacion, mencionar brevemente algun problema encontrado
	
	problemas de performance
	
	testing y escritura de tesis
	
	\subsection {Organizaci�n del documento}
	
	mencionar tdas las secciones con una mini descripcion

\section {Estado del Arte}

	\subsection {C�lculo de estructuras}
	
		\subsubsection {C�lculo implicados}
		aca hay que hablar el problea, les mandamos un mail paraver donde podemos leer mas
		
		\subsubsection {IETFEM}
		hablar de como resuelve el ietfem los problemas de arriba
		
		beneficios del ietfem, resuelve probelmas complejos como los comerciales
		
		carencias del ietfem, mencionar lso problemas un pocomas en detalle, mas que nada haciendo referencia a que a pesar de que soluciona probleas complejos eficientemente, asi como esta es muy dificil de usar 
		
		\subsubsection {Herramientas comerciales}
		
		Hablar de que centramos la investigacion en 2 grandes productos axis y sap 2000
		
		hablar de cada uno, quien lo desarrolla, si se puede en que esta hecho
		
		hablar del impacto mundial, que problemas resuelve, hablar de las cosas que nos gustaria poner en nuestro programa, cual nos gustaria evitar o mejorar
		
	\subsection {Desarrollo 3D}
	
		\subsubsection {OpenGL}
		
		investigar y hablar de open gl, que es, en que se usa, etc...
		
		hablar de porque no la elegimos, dificultad de uso, poca experiencia, etc...
		
		\subsubsection {Java 3D}
		
		lo mismo que al anterior
		
		\subsubsection {WebGL}
		
		lo mismo que el anterior
		
		hablar ademas de que al descubrir esta opcion se nos desperto la idea de hacerlo eb
		
		\subsubsection {Otras herramientas}
		
		hablar de otras herramientas de escritorio que hayamos visto, no se que mas poner aca
		
	\subsection {Desarrollo 3D en la Web}
	
		\subsubsection {HTML5 - Canvas}
		
		que es html5
		
		Hablar de las facilidades que da el html5 para cosas 3d mediante el canvas
		
		posibles conexiones entre ebgl y canvas
		
		\subsubsection {Librer�as para desarrollo 3D}
		hablar de las libreirias que vimos sobre ebgl, threejs y las otras que estuvios viendo que no me acuerdo
		
		porque elegimos threejs y que beneficios se tienen
		
		hacer un analisis si se pueden cumplir los objetivos con esta tecnologia
		
		\subsubsection {Interacci�n con el usuario}
		como pretendemos que sea la interaccion con el usuario, basandonos en los programas comerciales y lo que ofrece la eb
		
		plantearse cambiar el titulo de esta seccion para abarcar mas contenido
		
	\subsection {Informaci�n complementaria}
	
		\subsubsection {Investigaci�n sobre proyectos similares en Am�rica Latina}
		
		buscar proyectos similares en internet, en america latina y el mundo, y compararlos
		
		comparar lo encontrado con ietfem, y rematar se�alando que es el primer proyecto de este tipo en sudamerica
		
		\subsubsection {Herramientas de c�lculo de estructuras en la Web}
		
		Investigar si existen
		
		en casod e que existan, hacer una mini comparacion con lo que seria ietfem eb
		
		mencionar que no existan muchas herramientas y que tendria mucho potencial
		
\section {Organizaci�n del trabajo}

	\subsection {Alcance}
	mencionar las tareas que nos gustaria que el prgama haga, separando en 3 categorias
	
	entrada de datos: dibujado de nodos, barras, grillas, fueras, apoyos, importacion, abrir/guardar
	
	procesamiento de datos y comunicacion: generar textos para el core leer salida del core
	
	resultados: ver la deformada, colorear segun fueras, escalamiento, comparacion de resultados, etc
	
	hablar de que en el alcance pretendemos en general que se puedan resolver problemas del curso deelasticidad
	
	\subsection {Metodolog�a de trabajo}
	
	hablar de como nos repartimos la investigacion
	
	hablar de como dise�amos la app, que proceso de eleccion hicimos con las tecnologias
	
	como nos repartimos la programacion
	
	la escritura de la tesis
	
	
	\subsection {Estimaci�n y esfuerzo efectivo}
	
	para cada etapa mencionada, investigacion analisis, dise�o, implementacion, testing, tesis hacer diagramas y estimacinoes de tiempo mas o menos a la bartola

\section {Presentaci�n de la soluci�n}

	\subsection {An�lisis y relevamiento de requerimientos}
	
	hablar de las reuniones que tuvimos en la que explicaron el problema y fuimos deduciendo los casosde uso, mencionarlos
	
	hacer una lista de requerimientos
	
	referenciar los casos de uso en el anexo
	
	
	\subsection {Dise�o de la soluci�n}
	
		\subsubsection {Decisiones tomadas}
		
		HABLAR DE PORQUE LO HICIMOS EB Y LOCAL
		
		hablar de las reuniones interminables sobre el servidor y sus preocupaciones por el mntenimiento a futuro y otras complicaciones
		
		hablar brevemente de electron como posible alternativa
		
		hablar de las decisiones de no hacer grandes deformaciones, de hacer solo reticulados de lso atributosque no tomamos en cuenta, etc
		
		
		\subsubsection {Dise�o final}
		hablar de que dads las condiciones decidimos hacerlo eb pero manteniendo todo estatico, para que despues no haya problemas de compatibilidad
		
		mostrar un diagrama de dise�o
		
		hablar de cada modulo
		
	\subsection {Arquitectura}
	
		mostrar la arquitectura mediante diagramas de distribuciond e componentes y fisicas
		
		mostrar ademas como quedaria en version servidor
		
		distribucion de componentes: hablar de cada componente: manejo del espacio, del modelo, edicion de puntos, de lineas etc
		
		
	
	\subsection {Tecnolog�as y herramientas utilizadas}
	
		\subsubsection {HTML5 - Javascript - CSS3}
		
		dedicar un parrafo a cada una y como lo usamos particularmente en nuestro proyecto
		
		\subsubsection {Bootstrap}
		
		que es y como lo utiliamos en nuestro proyecto
		
		que mejoras se tienen con respecto a no utiliarlo
	
		\subsubsection {AngularJS}
		lo mismo que arriba
		
		\subsubsection {ThreeJS}
		mencionar que es la libreria principal
		
		como funciona, habar que etsa sobre ebgl, que funcione en un canvas, que se programa medante javascript, etc
		
		enque lo usamos en el proyecto
		
		mencionar y mostrar otros proyectos con three
		
		\subsubsection {Electron}
		mencionr que nosparecio esencial para la prolijidad del proyecto "local", ya que si aspiramos a que el estudiante quelo usa no tenga que lidiar con cosas de cimpuacion, seria contraproducente que tenga que abrir un html pelado en el navegador, donde puede tener problemas de compatibilidad con navegadores diferentes, etc
		
	\subsection {Manejo del espacio 3D}
	
		\subsubsection {Eventos de usuario}
		como definimoslo que puede hacer el usuairo
		
		los eventos definidos para el mouse y latecla de escape
		
		que componentes se comunican directamente con las acciones de usuario y describir "el camino" que realia cada una de ellas
		
		\subsubsection {Adici�n, sustracci�n y transformaci�n de objetos}
		
		hablar de la escena, como se agregan o modifican objetos
		
		como se conforma un objeto, atributos relevates
		
		hablar del renderiado
		
		acciones que se pueden hacer desde el programa
		
		\subsubsection {Manejo de la c�mara}
		
		hablar de orbit controls
		
		hablar de elcambio de ejes de coordenadas para que el  queda arriba
		
		hablar como se mueve la camara tratar de ver el js asqueroso y deducir matematicamente como se mueve con respecto al espacio 3d para dibujarlo o mostrar un diagramita
		
		\subsubsection {Trazado de rayos e intersecciones con objetos}
		
		hablar de como resolvimos la interseciion del 'click' con el rayo de la camara para seleccioar objetos
		
		hacer formulacion matematica
		
		hablar de las cosas que three automatia y de como qedo resuelto
		
		que compontes se enargan de esto
		
		\subsubsection {Performance}
		
		hablar de los probelmas que encontramos al probar la torre eiffel
		
		describir que se investigo y se hicieron camnios
		
		hablar de los geometry de los objetos
		
		de sacar a seleccion al hacer hover
		
		de otras performances que se hicieron
		
		hablar finalmente de otras mejoras que sepodrian hacer pero resultarian innecesarios porque serian muy complejos par este proyecto en el que los estudiantes nunca vana  hacer una estructura tan grande
		
	\subsection {Manejo de datos}
	
	hablar de como es la estructura que guardamos
	
		\subsubsection {Entrada de informaci�n (dibujado e importaci�n)}
		que informacino guardamos de cada elemento y porque
		
		en que momento agregamos cosas al mdoelo
		
		hablar del dibujado
		
		hablar de la importacion(como reemplaamos el modelo)
		
		hablar de abrir y guardar proyectos(como reemplaamos el modelo)
		
		\subsubsection {Mantenimiento de la estructura durante el proceso de dibujado}
		aca hablamso de como manipulamos los objetos: que s epueden modificas(agregar propiedades), o eliminar
		
		hablar de mantener la consistencia, al abrir un modelo nuevo, guardar, abrir, etc...
		
		\subsubsection {Almacenamiento de la estructura}
		hablar de que el modelo se va guardando en una variable javascript
		
		hacer un estudio sobre que tan eficiene=te es y si la memoria del navegador 'da' para almacenar algo asi
		
		mostrar un mini ejemplo y exactamente qu� se guarda
		
		\subsubsection {Salida de Datos}
		hablar de lo que se genera desde la ui
		
		como se genera, proceso que ace el usuairo para generarlo
		
		validaciones que se toman en cuenta
		
		como manipulamos el modelo ara generar el txt
		
	\subsection {An�lisis de resultados del Core}
	
		\subsubsection {Generaci�n de resultados}
		hablar sobre qu� genera el core
		
		mencionar las cosas que agrega el texto
		
		\subsubsection {Introducci�n de datos en la UI}
		
		como se ingresan
		
		como se modelan y almacenanesos datos - deformedmodel
		
		\subsubsection {Visualizaci�n}
		
		qu� se ve
		
		hablar de las opciones que se tienen
		
		como hicimos el sitcheo entre deformada en indeformada
		
		como hicimos el escalamiento
		
		como hicimos el colorado

\section {Resultados obtenidos}

	\subsection {Comparaci�n IETFEM con y sin UI}
	
		\subsubsection {An�lisis del impacto en la usabilidad}
		hablar de opinion de estudiantes, posiblemente en la idm
		
		mostrar unto por punto en que aspectos se mejoraron
		
		\subsubsection {An�lisis del impacto en el tiempo de ejecuci�n}
		usar ietfem viejo y nuevo y calcular tiempos
		
		ver si es muy dificil hacer un mini servidor para hacer una comparaciond de tiempos mejor
		
	\subsection {Casos de prueba}
	
		\subsubsection {Estudio de casos de peque�o porte (Torre peque�a)}
		hablar de que comenamos con ese
		
		se utilio para realiar la primera integracion con el core
		
		sedescurbrieron probelmasde ejes y se resolvieron facilmente
		
		\subsubsection {Estudio de casos de mediano porte (Gr�a)}
		
		se intento realiar el caso inicialmente como una prueba de stress
		
		no se encontraron problemas
		
		cuando se constato que funcionaba bien, se decidio utiliar un caso mas grande
		
		utiliado en la idm para mostrar le funcionamiento
		
		hablar de que ya se considera exitoso que funcione bien para la grua ya que lso estudiantes nunca van a hacer algo tan grande
		
		\subsubsection {Estudio de casos de gran porte y pruebas de stress (Torre Eiffel)}
		comentar que se decidio hacer latorre eiffel para ver como respondia el programa
		
		hablar del trabajo de la perfomrmance y memoria
		
		cuanto se mejoro luego de los arreglos
		
		importancia de que ande 'perfecto' ya que es un caso inalcanable

\section {Conclusiones y trabajo futuro}

	\subsection {Conclusiones}
	hablar si las estimaciones y el esfuerxo fueron acertados
	
	si se cumplieron los objetivos
	
	evaluar la herramienta
	
	\subsection {Trabajo a futuro}
	
		\subsubsection {Trabajo en el motor}
		que se puede agregar en el motor
		
		porticos
		
		osibilidad de migrar a otro lenguaje e integrar en un solo proyecto con la ui
		
		\subsubsection {Trabajo en la interfaz}
		agregar cosas que ya se pueden hacer enel core
		
		mejoras de performance
		
		otros 'chiches' que tienen programar comerciales
		
		delegar responsabilidades a aotra aplicacion
		
		\subsubsection {Despliegue de la aplicaci�n}
		hablar del servidor
		
		como funcionaria con servidor y porque no se hio asi
		
		donde se podria alojar
		
		mejoras que implicari en el sistema

\section {Anexos}

	mini Manual de uso
	
	ejemplos de estructuras
	
	modelo de dominio
	
	casos de uso
	
	diagramas de flujo
	
	diagramas de arquitectura
	
	mas info threejs
	
	masinfo otros proyectos similares
	
	otras cosas XD
	
	
\end{document}